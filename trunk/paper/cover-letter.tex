\documentclass[IANC,11pt]{infletr}
\usepackage[british]{babel}
\phone{511739}
\signature{David C. Sterratt\\Daniel~Lyngholm\\David J. Willshaw\\Ian
  D. Thompson}
\email{david.c.sterratt@ed.ac.uk}
%\usepackage[top=1.5in,left=1in,right=1in,bottom=1in]{geometry}
\begin{document}

\begin{letter}{The Editor-in-chief\\
    PLoS Computational Biology
}

\opening{Dear Professor Bourne,}

We are submitting for your consideration a manuscript ``Standard
anatomical and visual space for the mouse retina: computational
reconstruction and transformation of flattened retinae''.  Flat-mount
preparations of retinae are used extensively in studies of the
development and function of the visual system, but comparisons between
different eyes are complicated by the variability of the incisions and
tears inherent in the dissections. The manuscript describes our new
algorithm to reconstruct flat-mounted retinae by morphing them onto a
standard retina, which allows flat-mounted retinae from multiple eyes
to be compared. The location in standard space of the optic disc, the
only overt landmark in the mouse retina, is used to estimate the
accuracy of the reconstructions. We also render the insertions of the
eye muscles into standard space as well as the retinal locations of
retinal ganglion cells that do not cross the midline (the ipsilateral
retinofugal projection).


Furthermore, we determine the transformation from retinal to visual
space and use this to examine the visual projections of some covert
retinal features. Axonal tracing techniques are used to define the
ipsilateral and contralateral retinofugal projections. Plotting the
former in visual space defines the extent and location of the
binocular visual field when the ipsilateral decussation line
corresponds to the vertical meridian. There is disagreement in the
literature about the exact distribution of cells expressing short
wavelength opsin (S-opsin). We use standard retinal space to resolve
this conflict and then use the projection into visual space to confirm
an earlier prediction that a retinal boundary in the distribution
reflects the horizontal meridian.

We expect that the reconstruction of flattened retinae into standard
retinal space and the subsequent transformation into visual space will
have a wide range of applications in visual neuroscience: applications
not limited to the mouse.

The code that implements the algorithm is being made available for the
reviewers as supplemental information, and if this manuscript is
accepted we will make it available in the R-Forge repository, as
indicated in the manuscript.

\closing{Yours sincerely,}

\end{letter}


\end{document}

% LocalWords: PLoS  Bourne midline retinofugal
