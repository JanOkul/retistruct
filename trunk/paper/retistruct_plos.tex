% Template for PLoS
% Version 1.0 January 2009
%
% To compile to pdf, run:
% latex plos.template
% bibtex plos.template
% latex plos.template
% latex plos.template
% dvipdf plos.template

\documentclass[10pt]{article}

% amsmath package, useful for mathematical formulas
\usepackage{amsmath}
% amssymb package, useful for mathematical symbols
\usepackage{amssymb}

% graphicx package, useful for including eps and pdf graphics
% include graphics with the command \includegraphics
\usepackage{graphicx}

% cite package, to clean up citations in the main text. Do not remove.
\usepackage{cite}

\usepackage{color} 

% Use doublespacing - comment out for single spacing
%\usepackage{setspace} 
%\doublespacing


% Text layout
\topmargin 0.0cm
\oddsidemargin 0.5cm
\evensidemargin 0.5cm
\textwidth 16cm 
\textheight 21cm

% Bold the 'Figure #' in the caption and separate it with a period
% Captions will be left justified
\usepackage[labelfont=bf,labelsep=period,justification=raggedright]{caption}

% Use the PLoS provided bibtex style
\bibliographystyle{plos2009}

% Remove brackets from numbering in List of References
\makeatletter
\renewcommand{\@biblabel}[1]{\quad#1.}
\makeatother


% Leave date blank
\date{}

\pagestyle{myheadings}
%% ** EDIT HERE **

% FIXME: Remove
\newcommand{\svninfo}{$ $Rev$ $, $ $Date$ $}
\markboth{\svninfo}{\svninfo}

%% ** EDIT HERE **
%% PLEASE INCLUDE ALL MACROS BELOW

%% FIXME: MUST REMOVE BEFORE SUBMITTING TO PLOS
%\usepackage{todo}
\usepackage{color}
\newcommand{\todo}[1]{{\color{red}[#1]}}

%% END MACROS SECTION

\begin{document}

% Title must be 150 characters or less
\begin{flushleft}
{\Large
\textbf{Reconstruction of flattened retinae}
}
% Insert Author names, affiliations and corresponding author email.
\\
David C. Sterratt$^{1,\ast}$, Daniel Lyngholm$^{2}$, David
Willshaw$^{1}$, Ian D. Thompson$^{2}$
\\
\bf{1} Institute for Adaptive and Neural Computation, School of
Informatics, University of Edinburgh, Edinburgh, Scotland, UK
\\
\bf{2} MRC Centre for Developmental Neurobiology, King's College
London, London, UK
\\
$\ast$ E-mail: david.c.sterratt@ed.ac.uk
\end{flushleft}

% Please keep the abstract between 250 and 300 words
\section*{Abstract}

\todo{Between 250 and 300 words; currently 261}

In the course of studying the function and development of the visual
system, cells in the retina are often labelled; for example, to
determine the distribution of various types of cell in the retina, to
assess the location of retrograde tracer or to measure the
distribution of a guidance molecule.  Following labelling the retina
is dissected and flattened, and the distribution of labels is measured
in the flattened retina.  The dissection requires a number of
incisions to be made and tears in the rim and incisions can
develop. This complicates analysis as some cells that were close
neighbours are now separated by the incisions.

We present a computational method that overcomes this problem by
reconstructing the 3D shape of the retina so that the positions of the
labels in the intact retina can be inferred. The input to the
algorithm is the line segments of the flattened retinal outline, with
incisions and tears marked up by an expert. The retinal outline is
split into triangular elements whose positions are then transformed so
that they lie on a partial sphere with the expected dimensions of the
intact retina.  The transformation is adjusted so as to minimise a
physically-inspired deformation energy function. Our validation
studies indicate that the algorithm is able to estimate the position
of a point on the intact retina to within 8$^\circ$ of arc.

We demonstrate three applications of the method: (1) the analysis of
retrograde tracer experiments; (2) the analysis of mosaics of retinal
cells; and (3) infering the distribution over the intact retina of
Ephs and ephrins.

% Please keep the Author Summary between 150 and 200 words
% Use first person. PLoS ONE authors please skip this step. 
% Author Summary not valid for PLoS ONE submissions.   
\section*{Author Summary}

\section*{Introduction}

There has been extensive study of the function and development of the
retina and the mapping of retinal ganglion cells to target regions
such as the superior colliculus or lateral geniculate nucleus. In this
field a common experimental procedure is to label cells in the retina
\emph{in vivo} and then dissect and flatten the retina to produce a
flatmount in which the distribution of labels can be measured. In
order to flatten the retina, a number of incisions have to be made,
and tearing can occur within the rim of the retina or the incisions.
However, mounting the retina disturbs the retinal geometry
significantly, which can be problematic in interpreting results
obtained from flatmounted retinae.

For example, the distribution of mosaics of various cell types has
been recorded in flatmounted retinae
\cite{WassBoyc91func,RaveEtal03dete}.  Various measures are used to
quantify the regularity of mosaics, but these are susceptible to the
existence of boundaries \cite{Cook96spat}, both at the rim of the
retina and those introduced by the incisions required during
dissection. These extra boundaries reduce the effective number of
cells that can be analysed, which could be significant in low density
mosaics.

In the study of topographic mapping, the locations of retrograde
tracers injected in the superior colliculus have been measured in
retinal flat mounts \cite{RebeEtal04rela,RashEtal05oppo}. The
distribution of tracer can be split across incisions and tears,
complicating nearest-neighbour analyses and in some cases making it
impossible to determine a mean location that lies within the outline
of the flattened retina.

To understand the mechanisms underlying topographic mapping, retinal
flatmounts labelled with markers for the EphA receptor
\cite{ChenEtal95comp} have been produced. The principal axis of
variation in the flat mount appears to be the naso-temporal axis.
flatmounts stained for EphB receptors have also been produced
\cite{BirgEtal00kina} and here the principal axis is appears to be the
dorso-ventral axis. In both cases, the pixel intensity has been
quantified along the principal axis but not throughout the rest of the
retina.  Together with the finding that the mapping from a flat screen
to the superior colliculus is approximately Cartesian
\cite{DragHube76topo} this apparent alignment of EphA and EphB
gradients along the naso-temporal and dorso-ventral axes of the retina
has led to the mapping from the retina to the colliculus being
described as Cartesian \cite{BeviEtal11gene}. However, this cannot be
the case as the surface of the intact retina is by no means a
Cartesian surface, which can be described using rectilinear
coordinates.  Rather it needs to be described using curvilinear
coordinates. Being able to describe the density of marker as a
function of location in the intact retina would foster a deeper
understanding of how the mapping process occurs.

% The locations of the labels are measured in the two-dimensional
% Euclidean space defined by the slide, but this coordinate system is
% not entirely satisfactory for at least two reasons. Firstly, the
% incisions and tears may splits clusters of labelled cells which were
% centred around the same point in the intact retina. The mean location
% of such a split cluster in the Euclidean plane may lie outside the
% retinal outline. Secondly, it is difficult to infer the location with
% respect to the retinal pole and the nasal, temporal, ventral and
% dorsal poles with any degree of accuracy.

The aim of the method presented in this paper is to infer the
coordinates in intact retinal space of points on a flattened
retina. The retina is approximated as a partial sphere, and the
coordinate system can be thought of as lines of latitude and longitude
on the globe.  The method requires that an expert mark up the cuts and
tears, and also that they indicate the position of at least one
landmark on the rim of the eye. The method has been implemented as a
program written in R, available to download at
\todo{download location}. The method is estimated to be able to infer
the location of a point on a flat-mounted retina to within $8^\circ$
of arc of its original location. We demonstrate that the method is
applicable to the problems outlined above, namely retrograde tracing
experiments, mosaic analysis and marker distribution.

% Results and Discussion can be combined.
\section*{Results}

\subsection*{The reconstruction algorithm}

In this section we give an overview of the reconstruction algorithm; a
more detailed description is contained in the Methods. The starting
point of the algorithm is the retinal outline
(Figure~\ref{fold-sphere:fig:method}A), in this example with some
labelled points (red, green and yellow circles) and an example of a
landmark (blue line), in this case the optic disk.  The first step is
for an expert to mark the locations of incisions and tears in the
outline and to mark the location of one of the poles
(Figure~\ref{fold-sphere:fig:method}B). The outline is then
triangulated and the incisions and tears are stitched together
(Figure~\ref{fold-sphere:fig:method}C). This mesh is then projected
onto a spherical surface, with the points on the rim all lying at the
same latitude (Figure~\ref{fold-sphere:fig:method}Di). Each edge in
the mesh is treated as a spring whose natural length is the length of
the corresponding edge in the flat mesh.  In the initial mapping onto
the spherical surface, the springs are compressed or expanded; in the
next step the springs are allowed to relax so as to minimise the total
potential energy contained in all the springs. This gives the
spherical mesh shown in
Figure~\ref{fold-sphere:fig:method}Ei. Finally, the locations of data
points and landmarks in the flat retina can be projected onto the
reconstructed retina, which can be represented using a polar plot
(Figure~\ref{fold-sphere:fig:method}Fi). The mean locations of groups
of data points on the sphere are computed using the Karcher mean (see
Methods).

An alternative representation is shown in
Figures~\ref{fold-sphere:fig:method}Dii, Eii and Fii, where lines of
latitude and longitude on the spherical retina have been projected
onto the flat structure. The jump between Dii and Eii illustrates the
improvement in the appearance of the mapping achieved using the energy
minimisation.

To assess the amount of residual deformation at the end of the energy
minimisation procedure, we plotted the length of the edge in the
spherical mesh versus the length of the edge in the flat mesh
(Figure~\ref{retistruct_plos:fig:examples}Ai). We visualise the
deformation by representing the log strain as coloured edge in the
flat mesh (Figure~\ref{retistruct_plos:fig:examples}Aii). The log
strain is defined as $\ln(l/L)$ where $l$ is the length of an edge in
the spherical representation and $L$ is the length of the
corresponding edge in the flat representation. For a spring at its
natural length the log strain is zero (represented by green), for a
compressed spring it is negative (blue) and for an expanded one
positive (red).

A measure of the overall goodness of reconstruction is the square root
of the length energy function:
\begin{equation} 
  \sqrt{E_\mathrm{L}} = \sqrt{\frac{1}{2|\mathcal{C}|\overline{L}}
  \sum_{i\in\mathcal{C}} \frac{(l_i - L_i)^2}{L_i}}
\end{equation} 
where $L_i$ is the length of the $i$th edge in the flat mesh, $l_i$ is
the length of the corresponding edge in the spherical mesh, and the
summation is over, $\mathcal{C}$, the set of edges. The mean length of
an edge in the flat mesh is $\overline{L}$ and the number of edges is
$|\mathcal{C}|$. This measure is constructed so as to be of a similar
order to the mean log strain, which in turn is of the same order as
the mean fractional deformation.  

We used our algorithm to reconstruct 224 retinae.
Figure~\ref{retistruct_plos:fig:examples}A shows the reconstruction
with the lowest energy measure $\sqrt{E_\mathrm{L}}=0.038$ and
Figure~\ref{retistruct_plos:fig:examples}B the example with the
highest $\sqrt{E_\mathrm{L}}=0.147$. The arrangement of the grid lines
in the example with lower error looks qualitative smoother and more
even than in the example with higher energy measure
(Figures~\ref{retistruct_plos:fig:examples}Aiii
and~\ref{retistruct_plos:fig:examples}Biii). The strain plot for the
retina with the lower energy
(Figure~\ref{retistruct_plos:fig:examples}Aii) indicates that almost
all edges are unstressed, whereas in the retina with higher energy
there are many more compressed and expanded edges. It can be seen that
the retina in Figure~\ref{retistruct_plos:fig:examples}B has a much
less distinct rim than in Figure~\ref{retistruct_plos:fig:examples}A,
and this makes it harder for the algorithm to create an even mapping,
as local roughness in the rim forces significant deformation of the
surrounding virtual tissue when projected back onto the sphere.

The distribution of the goodness measure is shown in
Figure~\ref{retistruct_plos:fig:summary}. The mean was 0.073, with a
median value of 0.0736. 

\subsection*{Estimate of errors of the reconstruction algorithm}

The goodness measure described in the previous section gives an
indication of how easy it is to morph any particular flattened retina
onto a partial sphere, but it does not indicate the error involved in
the reconstruction, i.e.\ the difference between the inferred
position on the spherical retina and the original position on the
spherical retina. The ideal method for estimating the error would be
to flatten a retina marked in known locations, and then compare the
inferred with the known locations. However, this proved to be
technically very difficult and so we tried two other methods of
estimating the error.

The first method is to look at the inferred locations of the optic
discs across a number of retinae. In mice, the optic disc is located
``rather precisely in the geometric center of the retina''
\cite{DragOlse81gang} though this has not, as far as we are aware,
been measured. The optic disc has been marked in XXX of our
flat-mounted retinae, and the distribution of the centres of the
inferred locations of these optic discs is shown in
Figure~\ref{retistruct_plos:fig:ods}. The mean latitude and longitude
of these optic discs is (-85.7$^\circ$, 83.4$^\circ$) with a standard
deviation of 8.0$^\circ$. This 4.3$^\circ$ away from the geometrical
centre of the retina, in reasonable agreement with the unquantified
observation that the optic disc is at the geometric centre of the
retina.  Given that some of this variability is biological, this
suggests that an upper bound on the accuracy of the reconstruction
algorithm is 8.0$^\circ$.

The second method is to compare the inferred lines of longitude with
the direction growth of retinal ganglion cell axons towards the optic
disc.  The growth of retinal ganglion cell axons towards the optic
disc is radial and highly-ordered
\cite{ErskThom09intr}. Figure~\ref{retistruct_plos:fig:superpose}
shows the superposition of an image a retina \cite{KeelEtal11neur}
with the grid lines inferred by the method. It can be seen that the
grid lines are parallel to the RGC axons for most of their way, though
there are some deviations close to the rim.

\todo{Comparison of MRI images and reconstructed retinae.}

\subsection*{Application: retrograde tracing}

\todo{Need to discuss this with Dan and Ian}

\subsection*{Application: mosaic properties?}

\todo{Could discuss this with Stephen}

\subsection*{Application: marker distribution in spherical coordinates}

The gradients of Ephs and ephrins in the retina are usually regarded
as running down the nasotemporal (N--T) of dorsoventral (D--V)
axes. The A family of Ephs is associated with the N--T axis and the B
family with the D--V axis. The level of the labels, as inferred from
hybridised RNA \emph{in situ} stains has been quantified along the
axes \cite{ChenEtal95comp,RebeEtal04rela}.  However, the levels of
marker molecules off the axes have not been quantified. This is
important to know both for understanding the origin of retinal
gradients and how the retinocollicular mapping is set up. 

Figure~\ref{retistruct_plos:fig:superpose} shows the EphA3 gradient
from the literature \cite{ChenEtal95comp} projected onto a polar
plot. \todo{It would be good to do a contour plot of this.} 

Figure~\ref{retistruct_plos:fig:marcetal96} shows the EphA, EphB,
ephrin-A and ephrin-B gradients from the literature \cite{MarcEtal96eph}
projected onto polar plots.
\todo{I need to check rendering and contouring in these images, but
  they give an idea of what might be possible.}

\section*{Discussion}

\subsection*{Accuracy and validity of the method}

As far as we are aware, a computational algorithm for reconstructing
retinae has never been described
before.\footnote{\todo{DW: But what about optometry?}} Our validation
studies suggest that the method is able to estimate the orignal
location of a point on a flatted to within 8$^\circ$ of arc. This may
be an overestimate, as it makes the implicit assumption that the
location of the optic disc is the same between different retinae.
There will be some biological variablility in the location of the
optic disc which will tend to contribute to the measured variability.

We now consider the potential sources of error in the method. The
first source of error is the marking up of the incisions and
tears. There is a great deal of variability in dissected retinae; in
some it is clear where the vertices of incisions and tears lie and in
others is much less so. The current method depends on human input to
mark the incisions and tears. This has the potential disadvantage of
leading to variations between different experts but is more flexible
in the case of retinae with more or more irregular incisions and
tears. 

The physical model assumed by the algorithm is also a potential source
of errer. The retinal tissue is modelled as a mesh of masses connected
by springs. This basic model is used in modelling deformable surfaces
in computer animation and clothes design
\cite{FanEtal98spri,MaCaEtal99flat,WangEtal02surf} and has been used
in models of soft tissue \cite{SkriDunc99real}.  This model should
ensure that neighbourhood relations are preserved -- nodes close in
the 2D structure should still be close in the 3D structure -- and
incorporates the notion of an elastic material. However, it does not
incorporate an explicit representation of tensile and shear forces.

A more physically-principled method of modelling deformation of
objects is offered by the finite element method (FEM), in which the
stress and strain of each triangular element in the mesh is derived as
a function of the deformation of the points of the element
\cite{ZienTayl00fini}. The stress and strain matrices represent
shearing within the material, and materials with varying degrees of
compressibility (as quantified by Poisson's ratio) can be modelled.
The FEM is used widely to describe properties of soft tissue in
simulations of surgery \cite{CartEtal05appl}. The FEM could be applied
fairly straightforwardly to the forward problem of how a retina with
known incisions deforms during flattening. However, it is not possible
to apply the FEM directly to the problem of mapping the flattened
retina onto the intact retina: this would require knowledge of the
stresses and strains in the flattened retina, whereas the only
information available is the outline of the retina and the location of
the optic disk.

The meshing algorithm might also lead to errors. A problem can arise
when two points on the outline are very close together. This leads to
a large number of very samll triangles being constructed, which can
lead to a tangle of many flipped triangles in the energy minimisation
phase and, ultimately, small areas of high strain in the reconstructed
outline. However, this problem can be elimiated by suitable
preprocessing of outlines to remove very short edges.

\subsection*{Insights gained from the method}

\todo{What we have learnt by using the method}

\subsection*{Extentions to the method}

Another improvement might be to automate the marking up of incisions
and tears. It would be straightforward to produce a method that would
work for the retinae in which the incisions and tears are clear, but
it is harder to envisage a method that would work for the more
irregularly-flattend retinae.

It would be possible to generalise the method to deal with a retina
modelled by any shape with axial symmetry. This would just require a
different energy function to be used in the minimisation
procedure. This might lead to more accurate reconstructions, though
would depend on knowing the geometry of the intact retina more
accurately than at present.


% The soft tissue is often described as a linear-elastic medium, in
% which the stress depends linearly of the strain, but it can also be
% modelled as a viscoelastic medium, in which the stress depends on
% the history of the strain in the material.

% the finite element method is based more strongly on the physics of
% materials and is regarded as superior \cite{CartEtal05appl}.

% In principle, the FEM allows parameters of the material to be measured
% and incorporated in the model in a straightforward way. Given enough
% data about the material properties of retinal tissue, the model could
% be more realistic than the spring-mass model used here.  Weighed
% against this, 

%  Also, the standard FEM does not allow for the rotation of
% elements, only their translation or expansion and shearing, though it
% would be possible to overcome this using singlar value decomposition.

% Nevertheless, it would be possible to use a modified version of the
% FEM, along with data (or assumptions) about the material properties of
% retinal tissue to model directly the action of the retinal being
% flattened following the dissection cuts. This would lead to a
% prediction of how a retina cut in a particular set of loci would look
% when flattened. Unfortunately, we do not have precise knowledge of
% where the cuts were made in the intact retina or how tears developed,
% so some sort of complex iterative approach would have to be taken.

% However, the problem of mapping between 2D and 3D surfaces appears in
% other situations, such as in clothes design and computer animation
% . Workers in this field have taken a similar approach to the one taken
% here and their work has informed the algorithm presented here.
% % Surfaces are
% % triangulated and the edges in the triangulation are treated as springs
% % within the elastic limit.
% The basic physical model of the material as a mesh of masses connected
% by springs. 

% 3D surfaces are mapped to 2D surfaces by first triangulating the 3D
% surface, then projecting the points to the 2D plane, and then moving
% the locations of nodes of the triangulation on the plane so as to
% minimise an energy function based on the differences between the
% lengths of edges in the 3D configuration and in the 2D
% configuration. A penalty term to the error function can be added to
% prevent nodes crossing over edges \cite{WangEtal02surf}. In the
% mapping from 2D surface to 3D surfaces, the effects of forces due to
% sewing together edges of cloth and of gravity may be incorporated
% \cite{FanEtal98spri}.


% \subsection*{Potential problems with the method}
% \label{fold-sphere:sec:appr-simil-probl}

% The bulk viscoelastic properties of bovine retinal tissue have been
% measured.  The Young's modulus of rabbit retina in varying stages of
% development, under the assumption that the retina had linear
% elasticity and a Poission ratio of 0.5 \cite{ReicEtal91deve}.  The
% Poission's ratio is an index of how much a material shrinks in the two
% dimentions perpendicular to an expansion. A perfectly incopressible
% matierial has an Poission's ratio of 0.5. They found that the modulus
% increased threefold between P2 and P15. They also found that the
% stiffness of the retinal tissue depended on its location within the
% retina. The centre of the retina was stiffer than the periphery. This
% was due to the thickness of the tissue, which was greater in the
% centre than in the periphery.  The elastic properties of bovine
% retinal tissue dominate over the viscous properties, and that the
% material has a Poisson's ratio of around 0.45 \cite{LuEtal06visc}.

% \subsection*{Alternative methods}
% \label{fold-sphere:sec:alternative-methods}

% In the process of developing the solution presented in this paper, a
% number of options were evaluated:

% \paragraph{Forward or backward model:} 


% It is conceivable that the result of a forward model could be compared
% to the flattened outline in question, and the difference between the
% two used to modify the locations of tears and cuts on the intact model
% retina and that this procedure could be iterated until the patterns of
% cuts and tears on the intact retina converged.

% However, because it is highly likely that the model is a
% simplification of material properties of the real retina, it would be
% very unlikely that the outline of flattened model retinal would match
% the actual flattened retina, and we would be left with the problem of
% how to match up the edges of the actual and model flattened
% retinae. It is also difficult to envision an algorithm for inferring
% the locations of the tears and cuts that would be efficient. For these
% reasons, we rejected an iterative forward-backward modelling approach
% in which the tear locations on the intact retina were inferred.

% In the simple backward approach undertaken in this paper, the stresses
% and strains in the flattened retina are all assumed to be zero. In the
% mapping onto the intact retina, the elements of the mesh become
% strained. This is clearly unphysical, but may be a reasonable
% approximation to the actual Physics because:
% \begin{enumerate}
% \item For small stresses and strains, if the nodes are mapped back
%   onto the flattened retina, the locations shouldn't change much
%   (evidence?).
% \end{enumerate}

% \paragraph{Finite element versus spring-mass}
% \label{fold-sphere:sec:finite-elem-vers}


% Nevertheless, it would be possible to attempt to try applying the
% finite element, initialised with unstressed elements, to the backward
% transformation. This would give an estimate of the positions of the
% locations of nodes on the intact retina. These nodal positions could
% then be used to estimate the stiffness matrix for each element,
% assuming that each element was unstressed on the intact retina. The
% deformation of this structure when flattened could then be estimated
% by fixing the locations around the rim and edges, and solving for the
% unknown positions. This would lead to each element being strained, and
% these strains could then be used to initialise a second mapping back
% on to the intact retina, which would hopefully give rise to a more
% accurate estimation of the nodal positions. 

% This method would have the advantages of greater physical plausibility,
% and would afford the opportunity to incorporate elastic properties
% which vary with distance from the centre of the retina. However, this
% would come at the cost of some increase in complexity.

% You may title this section "Methods" or "Models". 
% "Models" is not a valid title for PLoS ONE authors. However, PLoS ONE
% authors may use "Analysis" 
\section*{Materials and Methods}
\label{retistruct_plos:sec:materials-methods}

\subsection*{Retinae}
\label{retistruct_plos:sec:retinae}

\todo{Some information about the preparation of the retinae.}

\subsection*{Reconstruction algorithm}
\label{retistruct_plos:sec:reconstr-algor}

The reconstruction algorithm described here has been implemented in R
(http://www.r-project.org) and can be downloaded as an R package from
http://www.neuralmapformation.org/code/retistruct. It has been tested
under a GNU/Linux platform but should also work in MacOS or Windows
environments.

The steps of the reconstruction algorithm are illustrated in
Figure~\ref{fold-sphere:fig:method}. The raw data
(Figure~\ref{fold-sphere:fig:method}A) consists of the sequence of
points making up the outline, sets of data points and sequences of
connected points defining landmarks, such as the optic disk. The
reconstruction process then proceeds as follows:
\begin{enumerate}
\item The location of one of the poles and all the incisions and tears
  in the outline are marked up by an expert
  (Figure~\ref{fold-sphere:fig:method}B). Each tear is defined by a
  central point, referred to as the apex, and two outer points,
  referred to as vertices.  Tears within tears or incisions (for
  example tear 2 in Figure~\ref{fold-sphere:fig:method}B) can be
  marked up. The Retistruct package contains a graphical user
  interface that allows incisions and tears to be marked up quickly.
\item The retinal outline is triangulated using the conforming
  Delaunay triangulation algorithm provided by the Triangle package
  \cite{Shew96tria} such that there are at least 500 triangles in
  the outline (grey lines in Figure~\ref{fold-sphere:fig:method}C).
\item The tears and incisions are stitched automatically (blue and
  green lines in Figure~\ref{fold-sphere:fig:method}D). To do this,
  the length of each side of every tear is computed. The fractional
  distance of each point in each tear is then defined as the distance
  along the tear of that point from the apex divided by the total
  length of that side of the tear. For each point on one side of a
  tear a corresponding point is inserted at the same fractional
  distance along the opposing side. Tears within tears are dealt with
  by excluding the child tear when computing the fractional
  distance. At the end of the procedure there is a set of
  correspondences between two or, in the case of the vertices of a
  child tear, three points.
\item There is then an extra round of triangulation to incorporate the
  points that have been inserted into incisions and tears.
\item The points within each set of correspondences are merged and
  allocated positions on the 2D surface.
\item The triangulation points are then projected onto a sphere
  curtailed at latitude of $\phi_0$
  (Figure~\ref{fold-sphere:fig:method}D). The latitude is estimated on
  the basis of measurements from intact retinae of animals of the same
  age as the retina under reconstruction. The radius $R$ of the sphere
  is determined by the area of the flattened retina and $\phi_0$.
  Points on the rim of flattened retina are fixed to the rim of the
  curtailed sphere.
\item The optimal projection onto the curtailed sphere
  (Figure~\ref{fold-sphere:fig:method}E) is inferred by shifting the
  locations of the vertices on the spherical surface so as to minimise
  an error measure. This measure, $E$, comprises the sum of normalised
  squared differences between lengths of corresponding connections in
  flattened and spherical retina, and a penalty term that prevents the
  triangles from flipping over:
  \begin{equation}
    E = \frac{1}{2|\mathcal{C}|\overline{L}} \sum_{i\in\mathcal{C}} \frac{(l_i - L_i)^2}{L_i}  
    + \alpha\sum_{j\in\mathcal{T}} (A_j/\overline{A})^\nu f(a_j/A_j)
  \end{equation}
  where $L_i$ and $l_i$ are lengths of corresponding edges
  $i\in\mathcal{C}$ in the flattened and spherical retina
  respectively, $\overline{L}$ is the mean length of an edge,
  $|\mathcal{C}|$ is the number of edges, $\alpha$ is a constant, $f$
  is a penalty function to be defined below, and $A_j$ and $a_j$ are
  signed areas of corresponding triangles $j\in\mathcal{T}$ in the
  flattened and spherical retina, $\overline{A}$ is the mean of $A_j$,
  and $\nu$ controls how much larger triangles are penalised than
  smaller ones.  The signed area $a_i$ is positive for triangles in
  correct orientation, but negative for flipped triangles. The penalty
  function $f$ is a piecewise, smooth function that increases with
  negative arguments:
  \begin{equation}
    \label{retistruct_plos:eq:1}
    f(x) = \left\{
        \begin{array}{ll}
          -(x - x_0/2) & x < 0 \\
          \frac{1}{2x_0}(x - x_0)^2 & 0 < x <x_0 \\
          0 & x \ge x_0
          \end{array} \right.
  \end{equation}
  The parameter $x_0$ is set at 0.5. There is thus no penalty unless
  triangles have been squashed to less than 50\% of their size in the
  flattened retina.  The parameter $\alpha$ is set to be large enough
  to prevent flipped triangles without causing numerical problems with
  the optimisation. The length of each edge $l_i$ is computed from the
  formula for the central angle between its vertices:
  \begin{equation}
    \label{retistruct_plos:eq:2}
    l(\phi_1, \lambda_1, \phi_2, \lambda_2) =
    R(\cos\phi_1\cos\phi_2\cos(\lambda_1-\lambda_2) +
    \sin\phi_1\sin\phi_2)
  \end{equation}
  where $\phi_1$ and $\phi_2$ are the latitudes of the vertices and
  $\lambda_1$ and $\lambda_2$ are the longitudes.  The derivatives of
  $E$ with respect to $\phi_i$ and $\lambda_i$ are computed and used
  to minimise $E$. The energy surface appears to contain many local
  minima. Many procedures were tried to optimise the performance of
  the algorithm, as measured on a corpus of over 200 marked-up
  retinae. The best procedure found was to first turn off the area
  penalty by setting $\alpha=0$ and using a modified version of the
  FIRE algorithm \cite{BitzEtal06stru}. This did a good job of
  minimising $E_\mathrm{L}$, but left a number of flipped
  triangles. Another run with FIRE algorithm with $\alpha=8$ and
  $\nu=1$ got rid of a number of the flipped triangles, dealing
  preferentially with the biggest. This was followed by a run of the
  BFGS quasi-Newton method (as implemented in the R optim function)
  with $\nu=0.5$.
\item The locations of data points and landmarks on the sphere are
  determined (Figure~\ref{fold-sphere:fig:method}F). To do this, for
  each data point the barycentric coordinates within its containing
  triangle in the flat representation are determined.  The location of
  the point on the sphere is then found by projecting the point to the
  same set of barycentric coordinates in the corresponding triangle on
  the sphere. From this location in Cartesian 3D space, the spherical
  coordinates of the point are determined by projection of a line from
  the centre of the sphere through the point to the line's
  intersection with the sphere. This allows plotting of points in a
  polar representation. The procedure can be used in reverse to infer
  the locations of lines of latitude and longitude in the flat retina.
\item The Karcher mean of each set of points is determined
  (Figure~\ref{fold-sphere:fig:method}F). The Karcher mean of a set of
  points on the sphere \cite{Karc77riem,HeoSmal06form} is defined as
  the point $(\overline{\phi}, \overline{\lambda})$ that has a minimal
  sum of squared distances to the set of points $(\phi_i, \lambda_i)$:
  %% See also BergWerm06
  \begin{equation}
    \label{retistruct_plos:eq:3}
    (\overline{\phi}, \overline{\lambda}) = \mbox{arg min}_{(\phi,
      \lambda)} \sum_{i=1}^N l^2(\phi, \lambda, \phi_i, \lambda_i)
  \end{equation}
  where the function $l$ is defined in
  Equation~(\ref{retistruct_plos:eq:2}).
\end{enumerate}

The procedure of reconstructing a retina takes a few minutes on a
modern desktop computer.

% Do NOT remove this, even if you are not including acknowledgments
\section*{Acknowledgments}

The authors would like to thank Andrew Lowe, David Willshaw, Stephen
Eglen, Johannes Hjorth, and Michael Herrmann for their very helpful
comments throughout this work.

%\section*{References}
% The bibtex filename
\newcommand{\myshortjournaltitles}{}
\bibliography{nmf_morph}
%\bibliography{template}

\section*{Figure Legends}
%\begin{figure}[!ht]
%\begin{center}
%%\includegraphics[width=4in]{figure_name.2.eps}
%\end{center}
%\caption{
%{\bf Bold the first sentence.}  Rest of figure 2  caption.  Caption 
%should be left justified, as specified by the options to the caption 
%package.
%}
%\label{Figure_label}
%\end{figure}

\begin{figure}[!ht]
  % Example. e.g. GM114-4/R-CONTRA (One subtear + gap)
  % GM184-5/R-CONTRA (One subtear, butter gap)
  % GM263-1/R-CONTRA (perfect!)
  \includegraphics{retistruct-method}
  
  \vspace*{-4.54in}

  \mbox{\hspace{2.27in}{
      \includegraphics[width=2.27in]{final-projection}
      \includegraphics[width=2.27in]{initial-projection}
    }}

  \vspace*{2.27in}

  \caption{\textbf{Overview of the method.} \textbf{(A)} The raw data:
    a retinal outline (black), three types of data point (red, green
    and yellow circles) and a landmark (blue line). \textbf{(B)}
    Retinal outline with nasal pole (N) and tears marked up. Each pair
    of dark cyan lines connect the vertices and apex of the five
    tears. Note that tear 2 is a child of tear 1. \textbf{(C)} The
    outline after triangulation (shown by grey lines) and stitching,
    indicated by cyan lines between corresponding points on the tears.
    \textbf{(Di)} The initial projection of the triangulated and
    stitched outline onto a hemisphere. The strain of each edge is
    represented on a colour scale with blue indicating compression and
    red expansion. Tears are shown in cyan. \textbf{(Dii)} The strain
    plotted on the flat outline with lines of latitude and longitude
    superposed. \textbf{(Ei,ii)} As \textbf{Di,ii} but after
    optimisation of the mapping. \textbf{(Fi)} The data represented on
    a polar plot of the reconstructed retina. Mean locations of the
    different types of data points are indicated by filled
    diamonds. The nasal (N), dorsal (D), temporal (T) and ventral (V)
    poles are indicated. Tears are shown in cyan. \textbf{(Fii)} Data
    plotted on the flat representation.
    \todo{Figures under D, E and F need to be grouped.}
    \todo{Arrows from
      A$\rightarrow$B$\rightarrow$C$\rightarrow$D$\rightarrow$E$\rightarrow$F
      would indicate the progression. }}
  \label{fold-sphere:fig:method}
\end{figure}

\begin{figure}[!ht]
  \centering
  \fontfamily{phv}\bfseries\selectfont
  \begin{tabular}{llll}
    Ai & Aii & Bi & Bii  \\
  \includegraphics[width=0.24\linewidth]{images-2011-09-02/nAChR-B2_adult_GMB530_R-CONTRA-strain-lvsL}
  &
  \includegraphics[width=0.24\linewidth]{images-2011-09-02/nAChR-B2_adult_GMB530_R-CONTRA-strain}
  &
  \includegraphics[width=0.24\linewidth]{images-2011-09-02/C57BL6J_P0_ML_GM182-4_R-CONTRA-strain-lvsL}
  &
  \includegraphics[width=0.24\linewidth]{images-2011-09-02/C57BL6J_P0_ML_GM182-4_R-CONTRA-strain}
  \\
  Aiii & Aiv & Biii & Biv  \\
  \includegraphics[width=0.24\linewidth]{images-2011-09-02/nAChR-B2_adult_GMB530_R-CONTRA-flat}
  &
  \includegraphics[width=0.24\linewidth]{images-2011-09-02/nAChR-B2_adult_GMB530_R-CONTRA-polar}
  &
  \includegraphics[width=0.24\linewidth]{images-2011-09-02/C57BL6J_P0_ML_GM182-4_R-CONTRA-flat}
  &
  \includegraphics[width=0.24\linewidth]{images-2011-09-02/C57BL6J_P0_ML_GM182-4_R-CONTRA-polar}
  \end{tabular}

  \caption{\textbf{Examples of reconstructed retinae.} \textbf{(A)}~An
    example of a good reconstruction.  \textbf{(Ai)}~Plot of length of
    edge on the sphere versus length of edge on the flat retina. Red
    indicates an edge that has expanded and blue a link that has been
    compressed.  \textbf{(Aii)}~The log strain indicated using the
    same colour scheme on the flat retina. \textbf{(Aiii)}~The flat
    representation of lines of latitude and longitude with data
    points. \textbf{(Aiv)}~The polar representation with data
    points. \textbf{(B)}~An example of a bad reconstruction. Meaning of
    \textbf{(Bi--iv)} same as for \textbf{(A)}.}
  \label{retistruct_plos:fig:examples}
\end{figure}

\begin{figure}[!ht]
  \centering
  \includegraphics{retistruct-2011-09-02-LFIRST}
  \caption{\textbf{Histogram of reconstruction errors.}  Successful
    reconstructions of 224 retinae were made. The energy at the end of
    each reconstruction was recorded and the histogram of the square
    root of this energy is displayed here. For lower values the value
    of $\sqrt{E_\mathrm{L}}$ is similar to the mean percentage
    deformation.}
  \label{retistruct_plos:fig:summary}
\end{figure}

\begin{figure}[!ht]
  \centering
  \includegraphics{superposed-ods-sqrtE_0_1}
  \caption{\textbf{Inferred positions of optic discs.}}
  \label{retistruct_plos:fig:ods}
\end{figure}

\begin{figure}[!ht]
  \centering
  \includegraphics{KeelEtal11}
  \caption{\textbf{Superposition of retinal image and grid lines.}}
  \label{retistruct_plos:fig:superpose}
\end{figure}

\begin{figure}[!ht]
  \centering
  \includegraphics{ChenEtal95}
  \caption{\textbf{Mapping of image to polar representation.}
    \textbf{(A)} Flat-mount E8 chick retina stained for EphA3
    (Reproduced from \cite{ChenEtal95comp}). \textbf{(B)} Image of
    gradient mapped to polar representation. \todo{Need to get
      permission for this data.}}
  \label{retistruct_plos:fig:superpose}
\end{figure}

\begin{figure}[!ht]
  \centering
%  \includegraphics{MarkEtal96}
  [Figure to follow]
  \caption{\textbf{Mapping of image to polar representation.}  All
    images are of E15 retinae stained by \cite{MarcEtal96eph} for EphA
    \textbf{(A,B)}, ephrin-A \textbf{(C,D)}, EphB \textbf{(E,F)} and
    ephrin-B \textbf{(G,H)}.
    \todo{Need to get permission for this data.}}
  \label{retistruct_plos:fig:marcetal96}
\end{figure}


\section*{Tables}
%\begin{table}[!ht]
%\caption{
%\bf{Table title}}
%\begin{tabular}{|c|c|c|}
%table information
%\end{tabular}
%\begin{flushleft}Table caption
%\end{flushleft}
%\label{tab:label}
% \end{table}

\end{document}


% LocalWords:  Sterratt Nedergaard MRC MacOS Retistruct BFGS optim Karcher arg
% LocalWords:  Acknowledgments nmf naso EphB dorso Ei Fi Dii Eii Fii Ai Aii DW
% LocalWords:  Willshaw Eglen Herrmann llll Bii Aiii Aiv Biii Biv th flatmount
% LocalWords:  colliculus geniculate vivo center RG
