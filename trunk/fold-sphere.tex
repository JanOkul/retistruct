\documentclass{article}

\usepackage{xspace}
\usepackage{amsmath}
\usepackage{graphicx}
\usepackage{natbib}
\usepackage{todo}
\usepackage[british]{babel}
\usepackage[left=1in,right=1in,top=1in,bottom=1in]{geometry}

\newcommand{\EE}{\ensuremath{E_\mathrm{E}}\xspace}
\newcommand{\EA}{\ensuremath{E_\mathrm{A}}\xspace}
\newcommand{\ED}{\ensuremath{E_\mathrm{D}}\xspace}
\newcommand{\Ed}{\ensuremath{E_\mathrm{d}}\xspace}
\newcommand{\p}{\vec{p}}
\newcommand{\q}{\vec{q}}


\title{Inference of original retinal coordinates from  flattenend
  retinae}
\author{David Sterratt}

\begin{document}
\maketitle

\begin{abstract}
  In retrograde tracing experiments to determine the mapping of
  connections from the retina to the superior colliculus in mice, a
  small blob of dye is injected in the superior colliculus and allowed
  to diffuse retrogradely down the axons of retinal ganglion cells to
  their cell bodies in the retina. The retina is dissected and
  flattened, and the pattern of dye in cell bodies can be seen in the
  flattened retina.  In the process of flattening the retina, a number
  incisions are made, and the pattern of dye can cut across incisions,
  complicating analysis.  One way of simplifying the analysis would be
  to infer the position of the cell bodies in the spherical coordinate
  system of the intact retina.  
\end{abstract}

\section{Introduction}
\label{fold-sphere:sec:introduction}

The general idea behind this method is to image folding the flattend
retina onto a sphere (or ellipsoid) that has the same dimensions that
would be expected of the real eye. Tears in the retina will naturally
tend to close up as they are folded onto the sphere.  In order to
perform the folding, the method has to allow for some elasticity in
the retina, and also prevent the edges of the flattened retina from
overlapping with each other on the surface of the sphere. The method
as developed so far does incorporate elasticity, but the prevention of
overlap has not yet been implemented.

\section{Algorithm}
\label{fold-retina:sec:method}

\subsection{Stitching algorithm}
\label{fold-sphere:sec:stitching-algorithm}

\begin{figure}[tp]
  \centering
  \includegraphics{plots/stitch}  
  \caption{Stitching. The solid line is the outline of the retina. The
    red circled-points are ones that were originally on the rim. The
    blue lines are the + tears and the purple lines are the - tears.
    Green lines indicate corresponding verticies at the end of tears.
    Yellow lines indicate corresponding points along the tears. }
  \label{fold-sphere:fig:stitch}
\end{figure}

Figure~\ref{fold-sphere:fig:stitch} shows a flattened
retina which has had the stitching algorithm applied to it.  The edge
of the flattend retina is described by a ordered set of $N$ points:
$\mathcal{P} = \{\vec{p}_1,\dots,\vec{p}_N$\}.  Vectors $\vec{g}^+$
and $\vec{g}^-$ of forwards and backwards pointers are set up:
\begin{displaymath}
  g^+_i = \left\{
    \begin{array}{ll}
      i+1 & i < N \\
      1   & i = N
    \end{array}\right.
  \quad
  g^-_i = \left\{
    \begin{array}{ll}
      i-1 & i > 1 \\
      N   & i = 1
    \end{array}\right.
\end{displaymath}
Cuts and tears marked up by an expert.  Each of the $M$ cuts or tears
$j$ is defined by a common apex at the point indexed by $A_j$, a
vertex in the forwards direction (indexed by $V^+_j$) and in the
backwards directions (indexed by $V^-_j$).  Pairs of verticies will
correspond to each other in the morphed retina, and this relationship
is indicated by a correspondence vectors $\vec{h}^+$ and $\vec{h}^-$:
\begin{displaymath}
  h^+_i =  \left\{
    \begin{array}{ll}
      i & \mbox{ if } i \not\in \{V^-_1,\dots, V^-_M\} \\
      V^+_j  & \mbox{ if } \exists j: i = V^-_j
    \end{array}\right.
  \quad
  h^-_i =  \left\{
    \begin{array}{ll}
      i & \mbox{ if } i \not\in \{V^+_1,\dots, V^+_M\} \\
      V^-_j  & \mbox{ if } \exists j: i = V^+_j
    \end{array}\right.
\end{displaymath}
At this stage, a correpondence vector $\vec{h}$ is initialiased to be
the same as $\vec{h}^+$.

The set of points in each tear is determined using the function
$\mathrm{path}$:
\begin{displaymath}
  \mathrm{path}(i, j, \vec{g}, \vec{h})  = \left\{ 
  \begin{array}{ll}
    \{i\} & \mbox{ if } i = j \\
      \{i, \mathrm{path}(g_i, j, \vec{g}, \vec{h})\} & \mbox{ if } i \ne j, h_i=i \\
      \{i, \mathrm{path}(h_i, j, \vec{g}, \vec{h})\}    & \mbox{ if } i \ne j, h_i\ne i \\
    \end{array}\right.
\end{displaymath}
This allows us to write:
\begin{displaymath}
  \mathcal{T}^+_j  = \mathrm{path}(A_j, V_j^+, \vec{g}^+, \vec{h}^+) \quad 
  \mathcal{T}^-_j  = \mathrm{path}(A_j, V_j^-, \vec{g}^-, \vec{h}^-)
\end{displaymath}
It is useful to determine the set of points $\mathcal{R}$ on the rim
of the retina.
\begin{displaymath}
  \mathcal{R} = \{1,\dots,N\} 
  \setminus (\mathcal{T}^+_1 \setminus V^+_1) 
  \setminus (\mathcal{T}^-_1 \setminus V^-_1)  
  \dots 
  \setminus (\mathcal{T}^+_M \setminus V^+_M)
  \setminus (\mathcal{T}^-_M \setminus V^-_M)
\end{displaymath}

The function $\mathrm{pl}$ defines the path length from one point to another
point:
\begin{displaymath}
  \mathrm{pl}(i, j, \vec{g}, \vec{h}, \mathcal{P}) = \left\{ 
    \begin{array}{ll}
      0 & \mbox{ if } i = j \\
      |\vec{p}_i-\vec{p}_{g_i}| + \mathrm{pl}(g_i, j, \vec{g}, \vec{h}, \mathcal{P}) & \mbox{ if } i \ne j, h_i=i \\
      \mathrm{pl}(h_i, j, \vec{g}, \vec{h}, \mathcal{P})    & \mbox{ if } i \ne j, h_i\ne i \\
    \end{array}\right.
\end{displaymath}
For each tear, the length of each side of the tear is computed:
\begin{displaymath}
  S^+_j = \mathrm{pl}(A_j, V^+_j, \vec{g}^+, \vec{h}^+, \mathcal{P}) \quad 
  S^-_j = \mathrm{pl}(A_j, V^-_j, \vec{g}^-, \vec{h}^-, \mathcal{P})
\end{displaymath}
Then for each point on one side of the tear, a new, corresponding,
point, with the index $n=N+1$ is placed at the same fractional
distance along the corresponding tear. To do this, the distance of a
point $i\in{\mathcal{T}^+_j}\setminus A_j \setminus V^+_j$ along the +
tear is computed:
\begin{displaymath}
  s^+_{ji} = \mathrm{pl}(A_j, i, \vec{g}^+, \vec{h}^+, \mathcal{P}) \quad 
\end{displaymath}
The node $k$ in the corresponding tear is the node which has the node
with the highest fractional distance $s^-_{jk}=\mathrm{pl}(A_j, k,
\vec{g}^-, \vec{h}^-, \mathcal{P})$ along the - tear which is still below
$s^+_{ji}$. The location of the new point is
\begin{displaymath}
  \vec{p}_n = (1-f)\vec{p}_k + f\vec{p}_{g^-_k}
\end{displaymath}
where
\begin{displaymath}
  f = \frac{s^+_{ji}S^-_j/S^+_j-s^-_{jk}}{s^-_{jg^-_k}-s^-_{jk}}
\end{displaymath}
The correspondnaces vector is updated so that $h_i = n$. The vector of
forward pointers is updated so that
\begin{displaymath}
  \begin{array}{ll}
    g^-_n = k     & g^+_n = g^+_k \\
    g^-_{g^+_k} = n & g^+_k = n 
  \end{array}
\end{displaymath}
A similar procecudure is carried out for the $-$ tear.

% The stitching algorithm uses this information to create
% correspondances between points on cuts and tears using an algorithm
% that

\subsection{Triangulation of flattened retina}
\label{fold-sphere:sec:triang-flatt-retina}

\begin{figure}[tp]
  \centering
  \includegraphics{plots/mesh}
  \caption{Triangular mesh}
  \label{fold-sphere:fig:mesh}
\end{figure}

Triangulation proceeds in a number of steps. 
\begin{enumerate}
\item $N_\mathrm{rand}=1000$ attempts are made to lay random points
  within the rectangle bounded by the minimum and maxiumum $x$ and $y$
  coordinates of members of $\mathcal{P}$.  Points which are outwith
  the retinal outline, as determined using the \texttt{InPoly}
  algorithm \citep{ORou98comp}, are rejected. Points which are less
  than $d=200$ from existing points are also rejected.
\item A Delaunay Triangulation of $\mathcal{D}$ the set of points
  $\mathcal{P}$, is formed. Triangles in $\mathcal{D}$ whose centres
  are outwith the retinal outline are excluded from $\mathcal{D}$.
\item From the triangluation $\mathcal{D}$ the set $\mathcal{C}$ of pairs of
  indicies representing line segments is constructed.
\item For the morphing algorithm, it is essential that no line segment
  connects non-adjacent nodes in the rim $\mathcal{R}$. Such line
  segments may be created by the triangulation. To detect such line
  segments, each member of $\mathcal{C}$ is checked to see if it is a
  subset of $\mathcal{R}$ and that its ends are not adjacent, by
  using $g^+$ and $g^-$. For each line segment that is found, it is
  removed by
  \begin{itemize}
  \item removing the two triangles that it belongs to from the list
    $\mathcal{D}$
  \item Creating a new point at the centroid of the four verticies
    shared by these two triangles.
  \item Creating four new triangles, who all share the new point and
    the each of which has two verticies from the set of four
    triangles.
  \end{itemize}
\item The same procedure is applyed to the longest edge that is longer
  than $2d$ and repeated until no edges are longer than $2d$. At every
  stage the set of connections $\mathcal{C}$ is recomputed to match
  $\mathcal{D}$.
\end{enumerate}

\todo{Incorporate this old meshing stuff}

The triangulation comprises $M$ triangles and is described by an
$M\times3$ matrix $T$, where each row contains the indicies of the
points in the triangle in anticlockwise order.

The connections are defined by $C_{ij}$, a symmetric, binary-valued
matrix that defines if there is a connection between $i$ and $j$ and
$L_{ij}$ are the distances between grid points on the flattened
retina.

\subsection{Projection onto hemisphere}
\label{fold-sphere:sec:proj-onto-hemisph}

It is supposed that this grid is to be projected onto a sphere with a
radius appropriate for the area $A$ of the flattened retina. The
radius is
\begin{equation}
  \label{fold-sphere:eq:1}
  R = \sqrt{\frac{A}{2\pi\sin\phi_0+1}}
\end{equation}
where $\phi_0$ is the latitude at which the rim of the intact retina
would be expected.

The aim now is to infer the latitude $\phi_i$ and longitude
$\lambda_i$ on the sphere to which each grid point $i$ on the
flattened retina is projected.  It is proposed to achieve this aim by
minimising an energy function which depends on three components, an
elastic component $E_\mathrm{E}$, an area-preserving component
$E_\mathrm{A}$ and a dipole component $E_\mathrm{D}:$
\begin{equation}
  \begin{split}
  E(\phi_1,\dots,\phi_N,\lambda_1,\dots,\lambda_N) = & \\
  E_\mathrm{E}(\phi_1,\dots,\phi_N,\lambda_1,\dots,\lambda_N) 
  & + E_\mathrm{A}(\phi_1,\dots,\phi_N,\lambda_1,\dots,\lambda_N) 
  + E_\mathrm{D}(\phi_1,\dots,\phi_n,\lambda_1,\dots,\lambda_n)
  \end{split}
\end{equation}

\subsection{The elastic energy}
\label{fold-sphere:sec:elastic-force}

This component of the energy corresponds to the energy contained in
imageinary springs with the natural lenght of the distances in the
flattened retina, $L_{ij}$:
\begin{equation}
  \label{fold-sphere:eq:5}
  \EE  = \frac{1}{2} \sum_{i=1}^N \sum_{j=1}^N C_{ij} (l_{ij}-L_{ij})^2  
\end{equation}
where $l_{ij}$ is the distance between grid points on the sphere and
is given by the formula for the central angle:
\begin{equation}
  \label{fold-sphere:eq:2}
  l_{ij} = R\cos^{-1}(\cos\phi_i\cos\phi_j\cos(\lambda_i-\lambda_j) + \sin\phi_i\sin\phi_j)
\end{equation}
Minimising this energy function should lead to the distances between
neighbouring points on the sphere being similar to the corresponding
distances on the flattened retina.

In order to minimise the function efficiently, the derivatives with
respect to $\phi_i$ and $\lambda_i$ are found:
\begin{equation}
  \label{fold-sphere:eq:3}
  \begin{split}
    \frac{\partial \EE}{\partial\phi_i} = 
    \sum_j C_{ij} (l_{ij} - L_{ij})R
    \frac{\sin\phi_i\cos\phi_j\cos(\lambda_i-\lambda_j) - \cos\phi_i\sin\phi_j}
    {\sqrt{1-(\sin\phi_i\sin\phi_j +
        \cos\phi_i\cos\phi_j\cos(\lambda_i-\lambda_j))^2}} \\
    \frac{\partial \EE}{\partial\lambda_i} = 
    \sum_j C_{ij} (l_{ij} - L_{ij})R
    \frac{\cos\phi_i\cos\phi_j\sin(\lambda_i-\lambda_j)}
    {\sqrt{1-(\sin\phi_i\sin\phi_j + \cos\phi_i\cos\phi_j\cos(\lambda_i-\lambda_j))^2}}
  \end{split}
\end{equation}
A quasi-Newton-Raphson method is then used in R to achieve this
optimisation, and the resulting grid on the sphere is plotted in 3D
(Figure~\ref{fold-sphere:fig:test2}).

\subsection{The area-preserving energy}
\label{fold-sphere:sec:area-pres-energy}

\subsection{The dipole energy}
\label{fold-sphere:sec:dipole-energy}

The total dipole energy is
\begin{equation}
  \label{fold-sphere:eq:6}
  \ED = \sum_{j=1}^n \sum_{k=1, k\ne j, k\ne j+1}^n \Ed(\p_j,
    \p_{j+1}, \p_k)
\end{equation}
where $\Ed(\p_j,\p_{j+1}, \p_k)$ is the energy due to
the attraction of the line segment between adjacent points $\p_j$ and
$\p_{j+1}$ on the boundary and the point $\p_k$.

The dipole energy is derived by integrating the point energy function of two
points separated by $r$:
\begin{equation}
  \frac{1}{1 + (r/d)^2}
\end{equation}
over the line segment
\begin{equation}
  \label{fold-sphere:eq:7}
  \Ed(\p_1, \p_2, \p_3) = -\int_{s_1}^{s_2}  \frac{1}{1 + (r/d)^2} ds
\end{equation}
where $r = \sqrt{r_0^2 + s^2}$ and
\begin{equation}
  \label{fold-sphere:eq:8}
  r_0 = \frac{1}{R^2}(\p_1 \times \p_2) \cdot \p_3
\end{equation}
and 
\begin{equation}
  s_1 = (\p_1 - \p_3)\cdot
  \frac{\p_2-\p_1}{|\p_2-\p_1|} \quad
  s_2 = (\p_2 - \p_3)\cdot
  \frac{\p_2-\p_1}{|\p_2-\p_1|}
\end{equation}
Solving~(\ref{fold-sphere:eq:7}) gives
\begin{equation}
  \label{fold-sphere:eq:9}
  \Ed(\p_1, \p_2, \p_3) = 
-\frac{d^2}{s_0}\left[\tan^{-1}\frac{s}{s_0}\right]_{s_1}^{s_2}
\end{equation}
where $s_0=\sqrt{d^2 + r_0^2}$

Differentiate:
\begin{equation}
  \label{fold-sphere:eq:10}
  \frac{\partial \ED}{\partial \p_i} = 
  \sum_{k=1, k\ne i, k\ne i+1}^n 
  \frac{\partial \Ed(\p_i, \p_{i+1}, \p_k)}{\partial \p_i}
  + \sum_{k=1, k\ne i-1, k\ne i}^n 
  \frac{\partial\Ed(\p_{i-1}, \p_{i}, \p_k)}{\partial \p_i}
  + \sum_{j=1, j\ne i-1, j\ne i}^n 
  \frac{\partial \Ed(\p_{j}, \p_{j+1}, \p_i)}{\partial \p_i}
\end{equation}

\begin{equation}
  \label{fold-sphere:eq:11}
  \frac{\partial \Ed(\p_i, \p_{i+1}, \p_k)}{\partial \p_i} =
  \frac{\partial \Ed}{\partial r_0} \frac{r_0(\p_i, \p_{i+1}, \p_k)}{\p_i}
\end{equation}

\begin{equation}
  \label{fold-sphere:eq:12}
  \frac{\partial r_0(\p_i, \p_{i+1}, \p_k)}{\partial \p_i} = 
  \frac{1}{R^2} \p_{i+1} \times \p_k \quad
  \frac{\partial r_0(\p_{i-1}, \p_{i}, \p_k)}{\partial \p_i} = 
  \frac{1}{R^2} \p_{k} \times \p_{i-1} \quad
  \frac{\partial r_0(\p_{j}, \p_{j+1}, \p_i)}{\partial \p_i} = 
  \frac{1}{R^2} \p_{j} \times \p_{j+1} 
\end{equation}

\begin{equation}
  \frac{\partial s_1}{\partial \p_1} = 
  \frac{\p_2 - \p_1}{|\p_2 - \p_1|} \quad
  \frac{\partial s_2}{\partial \p_1} = \vec{0}
\end{equation}

\begin{equation}
  \frac{\partial s_1}{\partial \p_2} = \vec{0} \quad
  \frac{\partial s_2}{\partial \p_2} =   \frac{\p_2 - \p_1}{|\p_2 - \p_1|} 
\end{equation}

\begin{equation}
  \frac{\partial s_1}{\partial \p_3} = 
  -\frac{\p_2 - \p_1}{|\p_2 - \p_1|} \quad
  \frac{\partial s_2}{\partial \p_3} = 
  -\frac{\p_2 - \p_1}{|\p_2 - \p_1|} \quad
\end{equation}

% \section{Application to retina}
% \label{fold-retina:sec:application-retina}

\begin{figure}
  \centering
  \includegraphics[width=0.6\linewidth]{test2-sphere}
  \caption{The same retina as in Figure~\ref{fold-sphere:fig:test1},
    but mapped onto a partial sphere with $\phi_0= 50^\circ$.}
  \label{fold-sphere:fig:test2}
\end{figure}

\begin{equation}
  \ED = \sum_{j=1}^n \sum_{k=1}^n \ED^{jk}
\end{equation}

\begin{equation}
  \begin{split}
  \frac{\partial \ED}{\partial \p_i} = 
  \sum_{k=1}^n & 
  \frac{\partial \ED^{ik}}{\partial r_0^{ik}}
  \frac{\partial r_0^{ik}}{\partial \p_i} +
  \frac{\partial \ED^{ik}}{\partial s_1^{ik}}
  \frac{\partial s_1^{ik}}{\partial \p_i} +
  \frac{\partial \ED^{ik}}{\partial s_2^{ik}}
  \frac{\partial s_2^{ik}}{\partial \p_i} \\
  &   
  + \frac{\partial \ED^{i-1k}}{\partial r_0^{ik}}
  \frac{\partial r_0^{i-1k}}{\partial \p_i} +
  \frac{\partial \ED^{i-1k}}{\partial s_1^{i-1k}}
  \frac{\partial s_1^{i-1k}}{\partial \p_i} +
  \frac{\partial \ED^{ik}}{\partial s_2^{ik}}
  \frac{\partial s_2^{i-1k}}{\partial \p_i} \\
  & 
  + \frac{\partial \ED^{ki}}{\partial r_0^{ki}}
  \frac{\partial r_0^{ki}}{\partial \p_i} +
  \frac{\partial \ED^{ki}}{\partial s_1^{ki}}
  \frac{\partial s_1^{ki}}{\partial \p_i} +
  \frac{\partial \ED^{ki}}{\partial s_2^{ki}}
  \frac{\partial s_2^{ki}}{\partial \p_i} \\
  \end{split}
\end{equation}



\section{Discussion}
\label{fold-retina:sec:discussion}

Figures~\ref{fold-sphere:fig:test1} and~\ref{fold-sphere:fig:test2}
show that the algorithm does project the flattened retina onto a
sphere. The edges of the retina at the outer ends of tears are closer
together on the sphere than they are in the flattened coordinate
system. However, towards the base of the tears, there is considerable
overlap between neighbouring ``flaps'' of the flattened retina. The
algorithm as it stands is therefore not sufficient to allocate every
point on the flattened retina a coordinate in the presumed sphere of
the intact retina.

The reason for these overlaps is that there is no component of the
energy function that prevents them occurring. I envisage that it would
be possible to put in a component of the energy function that provides
for short range repulsion between vertices on the edge of the
flattened retina and edges of the flattened retina. The energy
function might have longer range attraction. One possible
parametrisation would be the Lennard-Jones potential which is used to
model short range repulsion and longer range attraction between
molecules:
\begin{equation}
  \label{fold-sphere:eq:4}
  E(r) = 4\epsilon\left(\left(\frac{\sigma}{r}\right)^{12}-
    \left(\frac{\sigma}{r}\right)^{6}\right)
\end{equation}
where $r$ is the distance between molecules, $\epsilon$ defines the
depth of the potential well and $\sigma$ is the distance at which the
potential is zero.

A useful preliminary step before implementing this would be to improve
the algorithm that creates the triangular mesh over the flattened
retina, so that it completely fills the space by making links to the
edge of the retina and by laying links along the edge of the retina.

Were this approach to work, additional refinements might be possible,
such as adding a component of the energy function that draws together
points on either side of a rip which are correspond to each other with
high probability. Another possibility would be to allow the radius of
the sphere $R$ to vary within realistic bounds, to optimise the fit
further.

\bibliographystyle{natbib}
\bibliography{mystrings,my}

\end{document}

% LocalWords:  ij BP Raphson PDF
%%% Local Variables: 
%%% TeX-PDF-mode: t
%%% End: 
