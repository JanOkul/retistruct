\documentclass{article}

\newcommand{\svninfo}{$ $Rev$ $, $ $Date$ $}
\pagestyle{myheadings}
\markboth{\svninfo}{\svninfo}

\usepackage[a4paper,left=1in,right=1in,top=1in,bottom=1in,head=1in]{geometry}

\usepackage[british]{babel}

\title{The retinal mapping data and the nmf\_morph toolkit}

\begin{document}

\maketitle
\thispagestyle{myheadings}

\section{Data format}
\label{manual:sec:reading-data}

\begin{table}
  \begin{tabular}{ll}
    \hline
    \multicolumn{2}{c}{\textbf{FOR EACH BOUNDARY}} \\
    \hline
    \texttt{MAPNUM}   & id number of boundary \\  
    \texttt{MINLAT}   & min latitude      \\
    \texttt{MAXLAT}   & max latitude      \\
    \texttt{MINLON}   & min longitude     \\
    \texttt{MAXLON}   & max longitude     \\
    \texttt{LABLAT}   & latitude of label \\
    \texttt{LABLON}   & longitude of label\\
    \hline
    \multicolumn{2}{c}{\textbf{FOR EACH CELL}} \\
    \hline
    \texttt{XRED}     & $x$-coordinate if cell labelled red but not doubly\\
    \texttt{YRED}     & $y$-coordinate if cell labelled red but not doubly\\
    \texttt{XGREEN}   & $x$-coordinate if cell labelled green but not doubly\\
    \texttt{YGREEN}   & $y$-coordinate if cell labelled green but not doubly\\
    \texttt{XDOUBLE}  & $x$-coordinate if cell labelled doubly\\ 
    \texttt{YDOUBLE}  & $y$-coordinate if cell labelled doubly\\
    \texttt{XGRID}    & sample box cell is in \\
    \texttt{YGRID}    & sample box cell is in \\
    \texttt{PERIM}    & perimeter of cell \\
    \texttt{AREA}     & area of cell \\
    \hline
    \multicolumn{2}{c}{\textbf{ONE PER GRID BOX}} \\
    \hline
    \texttt{GRIDX}    & grid location of centre of sample box \\
    \texttt{GRIDY}    & grid location of centre of sample box \\
    \texttt{XGRIDCOO} & $x$-coordinate of centre of sample box \\
    \texttt{YGRIDCOO} & $y$-coordinate of centre of sample box \\
    \texttt{BOXSIZEX} & size (half width) of sample box in $x$-direction \\
    \texttt{BOXSIZEY} & size (half width) of sample box in $y$-direction \\
    \texttt{COMPLETE} & whether counting of sample has been completed\\
    \texttt{TOTALCEL} & total number of cells in this box\\
    \texttt{TOTALRED} & total number of red-only cells in this box\\
    \texttt{TOTALGRE} & total number of green-only cells in this box\\
    \texttt{TOTALDOU} & total number of double cells in this box\\
    \texttt{MEANPERI} & average perimeter of all cells \\
    \texttt{MEANAREA} & average area of all cells \\
  \end{tabular}
  \caption{Column headings of the \texttt{SYS.SYS} file.}
  \label{tab:data-format}
\end{table}

The data for each retina is stored in a separate directory. Within
each directory there are two files:
\begin{description}
\item[\texttt{SYS.SYS}] A table in SYSTAT format containing the
  coordinates of the red, green and doubly labelled cell bodies, and
  counts of labelled cell bodies within each grid box. The column
  headings shown in Table~\ref{tab:data-format}.  Each row of the
  table contains information only on a subset of the data, e.g.\ the
  coordinates of a red-labelled cell.
\item[\texttt{ALU.MAP}] A text file containing the coordinates of the
  map outline. The file comprises a number of sections, each starting
  with a single number, which is the number of lines to read in the
  next section. These lines have two numbers each, the $x$ and $y$
  coordinates of a vertex of the map outline.
\end{description}

\section{Reading in and displaying data}
\label{manual:sec:datafile-utils}

All code is to be found in the \texttt{trunk/src} directory. The R
program should be started from this directory in the following examples.

To read in data use the functions in \texttt{datafile-utils.R}. Here
is code to read in the data from a directory containing the
\texttt{SYS.SYS} and \texttt{ALU.MAP} files, as detailed above:
\begin{verbatim}
source("datafile-utils.R")
sys <- read.sys("/data/path/gm257-1-P8-C57BL6/")
map <- read.map("/data/path/gm257-1-P8-C57BL6/")
plot.sys.map(sys, map)
\end{verbatim}


\end{document}

%%% Local Variables: 
%%% TeX-PDF-mode: t
%%% End: 

% LocalWords:  MAPNUM MINLAT MAXLAT MINLON MAXLON LABLAT LABLON XRED YRED XGRID
% LocalWords:  XGREEN YGREEN XDOUBLE labeled YGRID PERIM GRIDX GRIDY XGRIDCOO
% LocalWords:  YGRIDCOO BOXSIZEX BOXSIZEY TOTALCEL TOTALRED TOTALGRE TOTALDOU
% LocalWords:  MEANPERI MEANAREA SYS SYSTAT ALU src datafile utils
